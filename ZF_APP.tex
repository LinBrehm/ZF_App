
\documentclass[a4paper,10pt]{article}
\usepackage[utf8]{inputenc}
\usepackage[ngerman]{babel}
\usepackage{geometry}
\usepackage{amsmath,amssymb}
\usepackage{enumitem}
\usepackage{fancyhdr}
\usepackage{titlesec}
\usepackage{multicol}
\usepackage{color}
\usepackage{listings}
\usepackage{graphicx}
\usepackage{xcolor}
\usepackage{soul}
\usepackage{setspace}
\usepackage{tabularx}



\geometry{top=1.8cm, bottom=1.8cm, left=0.8cm, right=0.8cm }
\pagestyle{fancy}
\fancyhead[L]{HSLU T\&A - APP Zusammenfassung}
\fancyhead[R]{Linus Brehm}
\fancyfoot[C]{\thepage}
\fancyfoot[R]{\today}

\setlength{\columnsep}{0.4cm}
\setlength{\parindent}{0pt}
\setlength{\parskip}{0.5em}   % Vertical space between paragraphs

\sethlcolor{yellow!50}

% Headings formatting
\titleformat{\section}{\large\bfseries}{}{0em}{}
\titleformat{\subsection}{\normalsize\bfseries}{}{0em}{}
% Syntax: \titlespacing{command}{left}{before-sep}{after-sep}
\titlespacing{\section}{0pt}{0.5ex}{0.2ex}
\titlespacing{\subsection}{0pt}{0.4ex}{0.1ex}

% Code formatting
\lstset{
  language=C,                         % Set to C or C++ depending on your code
  basicstyle=\ttfamily\small,         % Monospaced font
  backgroundcolor=\color{gray!8},   % Light gray background
  numbers=left,                  % Add line numbers on the left
  numberstyle=\tiny\color{gray},% Style of line numbers
  numbersep=5pt,                 % Distance from code
  commentstyle=\color{green!60!black},% Comments green
  keywordstyle=\color{blue},          % Keywords (data types, control words)
  stringstyle=\color{violet},         % Strings (optional)
  breaklines=true,                    % Auto-wrap long lines
  showstringspaces=false,             % Don't show spaces in strings
  morekeywords={uint8_t, uint32_t, bool, true, false}, % Custom keywords
}




%-----Hier beginnt das Dokument-----%
\begin{document}
\begin{multicols*}{2}

\section{Sequenz}
\underline{letztes Element} bestimmt Schleifenverhalten %highlight?
\begin{lstlisting}
if(statement_1, statement_2, statement_3);
while(statement_1, statement_2, statement_3);
for(init_1, init_2; check_1, check_2; calc_1, calc_2); 
\end{lstlisting}

\section{Deklarierung, Initialisierung \& Definition}
\textbf{Deklarierung:} macht Namen und Typ bekannt
\begin{lstlisting}
    int a;
\end{lstlisting}

\textbf{Initialisierung:} erste Zuweisung
\begin{lstlisting}
    a = 2;
    int b = 3;
\end{lstlisting}

\textbf{Definition:} reserviert Speicherplatz
\vspace{0.5em}

\subsection{Komplexe Deklaratoren}
\begin{enumerate}
\item Beim Bezeichner beginnen (von innen nach aussen)
\item \texttt{() und []} vor \texttt{*}
\end{enumerate}
\begin{lstlisting}
int *ap[10];    //Feld von int-Zeigern
int *fp(int);   //Funktion mit int-Ptr return
int (*pf)(int); //Ptr auf Funktion mit int arg
int *(*x[10])(int)
\end{lstlisting}
\begin{enumerate}
\item Array x von 10...
\item * Zeiger auf...
\item Funktionen...
\item mit int Argument
\item und int Zeiger als Rückgabewert
\end{enumerate}

\section{break, return, contiune, goto}
\underline{break} bricht Schleife und Switch-Case ab\\
\underline{return} springt aus der Funktion\\
\underline{continue} springt ans Ende der Schleife\\
\underline{goto} sprint zu angegebenem Label
\begin{lstlisting}
goto a;
int b = 5; //not executed
a: //jump here
\end{lstlisting}

\section{Typenspezifizierer}
void,char,int,short,...

\section{Typenqualifizierer}
\underline{const} Erstellt eine schreibgeschützte Variable\\
\underline{volatile} Weist den Compiler auf eine flüchtige Variable hin

\section{sizeof(Typenspezifizierer) $\rightarrow$ size\_t byte}
\begin{center}
  \begin{tabularx}{\linewidth}{lcc}
    Type & $ = $ & byte\\
    \hline
    uint8\_t, char, '3'  & $ = $ & 1 \\
    \hline
    uint16\_t & $=$ & 2\\
    \hline
    uint32\_t, int, float, pointer*, '3' + 3, 3.14f & $=$ & 4\\
    \hline
    uint64\_t, 3.1, double & $=$ & 8\\
    \hline
\end{tabularx}  
\end{center}

\vfill\null
\columnbreak
\section{Typengrössen C99}
\includegraphics[width = \linewidth]{Bilder/Typengroessen_c99.png}
\section{Speicherklasse}
\begin{lstlisting}
int a;
extern int b;
static int c;
void f(int d, register int e){
    auto int g;
    int h;
    static int i;
    extern int j;
    register int k;
}
\end{lstlisting}

\begin{center}
\small
\begin{tabular}{cccc}
\textbf{Name} & \textbf{Klasse} & \textbf{Bereich} & \textbf{Verknüpfung} \\
\hline
a & statisch  & Datei     & Extern \\
b & statisch  & Datei     & nDef   \\
c & statisch  & Datei     & Intern \\
d & automatisch & block   & keine  \\
e & automatisch & block   & keine  \\
g & automatisch & block   & keine  \\
h & automatisch & block   & keine  \\
i & statisch    & block   & keine  \\
j & statisch    & block   & nDef   \\
k & automatisch & block   & keine  \\
\end{tabular}
\end{center}

\underline{auto} Automatische Speicherdauer, Blockbereich \& keine Verknüpfung. Nur zulässig für Variablen innerhalb eines Blocks (ebenfalls Standard für jene).\\
\underline{static(ausserhalb)} Interne Verknüpfung, nicht von anderen Dateien sichtbar.\\
\underline{static(innerhalb)} Statische Speicherdauer, Variable wird nur einmal initialisiert und bleibt bestehen über mehrere Aufrufe.\\
\underline{extern} Nur Deklaration der Variable, wird an anderer Stelle definiert. So kann auf externe Variablen zugegriffen werden.\\
\underline{Funktionen} Können extern und static besitzen, mit derselben Funktion.

\section{Kontrollstrukturen}
\subsection{if-Schleife}
\begin{multicols*}{2}
\textbf{Normal}
\begin{lstlisting}
if(expression_1){
    statement_1;
    statement_2;
} else if(expression_1){
    statement_a;
    statement_b;
} else{
    statement_else;
}
\end{lstlisting}
\textbf{Kurz}
\begin{lstlisting}
if(expression_1)
    single_statement;
else if(expression_2)
    single_statement;
else single_statement;
\end{lstlisting}
\begin{lstlisting}
if(expression_1)?
    true_statement:
    flase_statement;
\end{lstlisting}
\end{multicols*}

\subsection{for-Schleife}
Wiederholt die Schleife bis check \underline{false} zurückgibt.
\begin{lstlisting}
for(init; check; update){/*loop*/}
\end{lstlisting}

\subsection{while-Schleife}
Wiederholt die Schleife bis check \underline{false} zurückgibt.
\begin{lstlisting}
while(check){/*loop*/}
\end{lstlisting}

\subsection{do-while-Schleife}
Führt die Schleife einmal aus und wiederholt sie bis check \underline{false} zurückgibt.
\begin{lstlisting}
do{/*loop*/} while(check)
\end{lstlisting}

\subsection{switch-case}
Ohne \underline{break} = fallthrough\\
Expression muss eine \underline{Ganzzahl} sein.\\
\underline{default} wird immer als letztes ausgeführt!
\begin{lstlisting}
switch(expression){
    case 3: expression; //fallthrough
    case 1:
       expression; break;
    case 5: expression; break;
    default: break;
}
\end{lstlisting}

\section{Typedef}
Mit typedef kann man neue Typen erstellen (oder existierende einen anderen Namen geben). Diese haben immer die Form:
\begin{lstlisting}
typedef type typeName_t
\end{lstlisting}
\subsection{Normal}
\begin{lstlisting}
typedef float komma_t;
\end{lstlisting}
\subsection{Pointer}
\begin{lstlisting}
typedef int *int_ptr;
int a = 5;
int_ptr pa = &a;
\end{lstlisting}
\subsection{String}
\begin{lstlisting}
typedef char *string_ptr;
string_ptr strg = "Hello World!"
printf("%s\n",strg);
\end{lstlisting}
\subsection{Funktionen}
\begin{lstlisting}
typedef int (function)(int,int);
function add;
int add(int a, int b){
    return a+b;
}
\end{lstlisting}
\subsection{Struct,Union,Enum}
\begin{lstlisting}
typedef struct { int val; } node_t;
node_t myNode = {.val = 10 };
printf("%i\n",myNode.val);
\end{lstlisting}
\begin{lstlisting}
typedef union {
   uint8_t a;
   uint8_t b;
} bytes_u;
bytes_u number = {.a = 8};
\end{lstlisting}
\begin{lstlisting}
typedef enum {
    EAST, NORTH, WEST, SOUTH
}dirct_e;
dirct_e dir = NORTH;
\end{lstlisting}

\section{Speicher}
\subsection{RAM Random Accsess Memory}
es kann auf jede Speicherzelle (Byte-Weise) direkt zuzugreiffen speichert Variablen, Stack, Heap
\begin{itemize}
    \item schnelles lesen \& schreiben
    \item Daten sind nicht persistent bei Stromunterbruch (flüchtig)
\end{itemize}
\textbf{Stack:} Stappel von Elementen (Werte, Funktionen, etc.; nach dem LIFO-Prinzip (Last In, First Out) → zuletzt hinzugefügte Element wird als erstes wieder entfernt\\
\textbf{Heap:} Speicherplatz für dynamische Speicherverwaltung (für Daten deren Grösse / "Lebensdauer"noch nicht bekannt sind) $\rightarrow$ malloc
\subsection{Flash}
nichtflüchtiger Speicher (behält Daten bei Stromunterbruch)
\begin{itemize}
    \item konstante Daten
    \item kann nur Blockweise beschrieben / gelesen werden
    \item wird typischerweise während Laufzeit nicht verändert beschrieben des Speicher $\rightarrow$ flashen
\end{itemize}
\section{Speicherzugriff}
\begin{center}
    \includegraphics[width = 0.5\linewidth]{Bilder/RAM.png}
\end{center}
$\Rightarrow$Datenbus ist bi-direktional\\
\textbf{lesen:}\\
CPU $\rightarrow$ Adrese and Adressbus\\
RAM $\rightarrow$ Wert an Datenbus\\
CPU $\rightarrow$ liest den Wert an der Adresse\\
\textbf{schreiben:}\\
CPU $\rightarrow$ Adrese and Adressbus\\
CPU $\rightarrow$ zu schriebende Wert an Datenbus\\
CPU $\rightarrow$ R-W $\Rightarrow$ W\\
RAM $\rightarrow$ Wert and Adresse\\
Memory Mapped I/O: I/O direkt mit dem Bus verbunden (sonst
über einen Port)


\section{Dynamischer Speicher}
\subsection{Malloc}
Wenn erfolgreich für n-Elemente Speicher reserviert worden ist, wird die 1.Adresse des Speicherblocks zurückgegeben. Falls nicht genügend Speicherplatz vorhanden ist, wird NULL zurückgegeben.
\begin{lstlisting}
void *malloc(size_t size);//Functionprototype
uint8_t *ptr = malloc(n* sizeof(uint8_t));
\end{lstlisting}
\subsection{Calloc}
Selbe Funktion wie malloc, jedoch muss die Anzahl und die Grösse separat mitgegeben werden und der \underline{Reservierte Speicher mit 0 Initalisiert.}
\begin{lstlisting}
void *calloc(size_t ammount,size_t size); 

uint8_t *ptr = calloc(n,sizeof(uint8_t));
\end{lstlisting}
\subsection{Realloc}
Erweitert den reservierten Speicher und gibt bei Erfolg die 1.Adresse des neuen Speicherblocks zurück, ansonsten NULL. Aus diesem Grund muss der \underline{Rückgabewert auf NULL überprüft werden.}
\begin{lstlisting}
void *realloc(void *ptr, size_t size);

uint8_t *tmp = realloc( ptr, n );
if(tmp != NULL) ptr = tmp; // Wichtig!!!
\end{lstlisting}
\subsection{Free}
Gibt den Speicher wieder frei. Der Pointer \underline{muss} auf die \underline{1.Adresse} zeigen. (Zeigerarithmetik nie mit ptr ausführen)
\begin{lstlisting}
void free(void *ptr);
if(ptr != NULL){ // Wichtig!!!
free(ptr);
}
\end{lstlisting}

\section{Typcast}
\subsection{Implizite Konvertierung}
Wird vom Compiler automatisch gemacht, erzeugt eine Warnung.
\begin{lstlisting}
int var = 'A' // from char to int
\end{lstlisting}
\subsection{Explizite Konvertierung}
Wird mit einem cast-Operator gemacht werden, die saubere Variante.
\begin{lstlisting}
int var = (int)'A'; from char to int
\end{lstlisting}

\section{Zeichenketten}
Strings können auf zwei unterschiedliche Varianten gespeichert werden. Die Ausgabe unterscheidet sich nicht, jedoch die Funktionsweise.
\subsection{Definition als Pointer}
Wenn ein String als Pointer abgespeichert wurde, ist der Bezeichner (prt) veränderbar. Jedoch können die \underline{Elemente nicht verändert} werden.
\begin{lstlisting}
char *str = "Hallo";
str = str + 2;  //Zeigt nun auf erstes 'l'
char ch = "Hallo"[1]; // ch = a (lesen)
str[0] = 'x';   //FEHLER (schreiben)
*(str+2) = 'x'; //FEHLER
sizeof(str);    // = 4 (32bit Adresse)
\end{lstlisting}
Zudem hat der String bei der Definition als Pointer beim Aufruf des sizeof-Operators die \underline{Grösse 4} (da es ein Pointer ist) und nicht die tatsächliche Länge des Strings.
\subsection{Definition als Array}
Wenn ein String als Array abgespeichert wurde,ist der Bezeichner (ptr) \underline{nicht veränderbar.} Jedoch können die Elemente verändert werden.
\begin{lstlisting}
char str[] = "Hallo"; // 'H','a',..,'o','\0'
str[0] = 'x'    // 'H' wird durch 'x' ersetzt
printf("%c",*(str+1)); // print 'a'
str = str + 2;  //FEHLER
sizeof(str);    // = 6 (Array + '\0')
\end{lstlisting}
Bei der Definition als Array ergibt sich bei Aufruf des sizeof-Operators die \underline{Grösse der Zeichenkette + 1}.
\subsection{2-Dimensionale char Arrays}
Mit dieser Methode müssen alle Zeilen dieselbe Länge haben, auch wenn die nicht gebraucht werden.
\begin{lstlisting}
//3 Namen (Zeilen) mit je 12 Zeichen (Spalten)
char names[3][12] = {"Urs","Maria","Tom"};
sizeof(names);  // = 36 (3 * 12 char)
\end{lstlisting}
\subsection{Array von char-Pointern}
Mit dieser Methode können Zeilen eine unterschiedliche Länge haben, somit ist dies speichereffizienter.
\begin{lstlisting}
// 3 Namen mit unterschiedlicher Anz. Zeichen
char *names[3] = {"Urs","Anna-Maria","Tom"};
sizeof(names)   // = 12 (3 * 32bit Adresse)
\end{lstlisting}

\section{Pointer}
\subsection{Pointer Grafisch}
\includegraphics[width = 0.4\linewidth]{Bilder/Pointer_grafisch.png}

\subsection{Zeigerarithmetik}
Zeigerarithmetik kann nur mit Array-Pointern im selben Array durchgeführt werden, da ansonsten undefiniertes Verhalten auftritt.\\
Um den Pointer eines Arrays zu erhalten, gibt es folgende Möglichkeiten:
\begin{lstlisting}
int *p1 = &a[2];//pointing at 3.element(2)
int *p2 = a;    //pointing at 1.element(0)
\end{lstlisting}
Mit diesen Zeigern kann man folgende Operationen durchführen:
\begin{enumerate}
    \item Differenz zwischen Pointern (p1 - p2)
    \item Ganzzahl Addition
    \item Ganzzahl Subtraktion
\end{enumerate}
\begin{center}
    \includegraphics[width = 0.8\linewidth]{Bilder/Zeigerarithmetik.png}
\end{center}
Die Operatoren * und $--$ lassen sich in derselben Art und Weise kombinieren wie * und $++$.

\subsection{Pointer auf Pointer}
Wird verwendet, um den Wert eines Pointers aus einer Funktion zu bearbeiten.
\begin{lstlisting}
int a = 3;
int *ptr1 = &a;
int **ptr2 = &ptr1;
printf("%i\n",**ptr2); // = 3
\end{lstlisting}

\subsection{Pointer auf Funktion}
Um eine Funktion einer Funktion als Parameter übergeben zu können, muss ein Pointer dafür angelegt werden. Somit zeigt der Pointer auf dieselbe Adresse wie die Funktion selbst.
\begin{lstlisting}
int add(int x, int y){retrun x+y;}
int (*foo)(int,int) = add;
printf("%i\n", foo(3,4));   // = 7

// array von funktionspointer
int(*fpa[4])(float) = {qux};
// Funktion for Funktionspointer im array
int qux (float);

int (*bum(void))(float);
*bum(void);// Funktion mit void Argument und Funktionspointer auf
int foo (float); // als Rueckgabewert
\end{lstlisting}

\subsection{const Pointer}
\begin{lstlisting}
[type] *ptr;      //Pointer mit Typ
const [type] *ptr;//Wert ist nicht aenderbar
[type] *const ptr;//Adresse ist nicht aenderbar
\end{lstlisting}

\section{Build}
\begin{enumerate}
    \item main.c $\rightarrow$ \textbf{Präprozessor} $\rightarrow$ Vorverarbeiteter Quellcode
    \item \textbf{Compiler} $\rightarrow$ Object-Datei, main.o
    \item \textbf{linker} $\rightarrow$ ausführbare Datei, program.axf
\end{enumerate}

\section{Präprozessor}
\subsection{\#include}
'code.h' muss in 'main.c' und 'code.c' inkludiert werden.
\begin{lstlisting}
#include <stdio.h>  //Systemdateien
#include <stdint.h>
#include <string.h> 
#include "board.h"  //lokale Dateien
\end{lstlisting}
\subsection{\#define}
Werden vom Präprozessor textuell ersetzt.\\
Einfache Makros werden verwendet, um Konstanten namentlich im Code zu verwenden.
\begin{lstlisting}
#define PI 3.141592
float area = PI*radius*radius;
\end{lstlisting}
Makrofunktionen haben Parameter, welche in die Funktion eingesetzt werden. Die Funktion muss immer mit genug \underline{Klammern} versehen sein um zufälliges Verhalten zu vermeiden.
\begin{lstlisting}
#define ADD(a,b) ((a) + (b))
#define MAX(x,y) ((x) > (y) ? (x) : (y))
int c = ADD(4*5,7);     
int max = MAX(t++,7);//t zweimal inkrementiert
\end{lstlisting}
\subsection{\#undef}
Mit \#undef kann ein \#define aufgelöst werden.
\subsection{Header-Guard}
Haltet den Präprozessor davon ab, eine Header-Datei mehrmals zu inkudieren.
\begin{lstlisting}
#ifndef _INCLUDE_H_
#define _INCLUDE_H_
/* ... include.h declereations here ... */
#endif
\end{lstlisting}
Wichtig zu beachten ist, das \#ifndef und \#ifdef nur auf \#define achtet und \underline{nicht auf Variablen}
\subsection{\#if}
Mit \#if, \#elif und \#endif kann eine Bedingte Kompilierung erzeugt werden.
\begin{lstlisting}
#if defined(WIN32)
    /* Code */
#elif defined(LINUX)
    /* Code */
#endif
\end{lstlisting}
\subsection{\# \& \#\# Operatoren}
\# Wandelt Makroargument in Zeichenkettenliteral um.
\begin{lstlisting}
#define PRINT_INT(n) printf(#n" = %d\n",n)
PRINT_INT(i/j);             //Makro Aufruf
printf("i/j"" = %d\n",i/j); //Praeprozessor
printf("i/j = %d\n",i/j);   //Nach Compiler
\end{lstlisting}
\#\# Fügt Tokens zusammen.
\begin{lstlisting}
#define paster(n) printf("id"#n" = %d", id##n)
paster(9);                     //Makro Aufruf
printf("id" "9" " = %d",id9);  //Praeprozessor
printf("id9 = %d",id9);        //Nach Compiler
\end{lstlisting}
\subsection{Aufgabe}
Der Präprozessor wird vor dem eigentlichen Kompilieren ausgeführt. Er verarbeitet sogenannte Präprozessor-Direktiven, die mit \# beginnen.
Er erstellt eine „Vorverarbeitete“ Version des Codes, z.B. ersetzt er \#include $<$stdio.h$>$ durch den tatsächlichen Inhalt der Datei.
\subsection{Typische build Fehler}
\textit{Fehlende Header-Dateien:} fatal error: myheader.h: No such file or directory\\
\textit{Makro-Probleme:} falsche oder rekursive Makrodefinitionen\\
\textit{Syntaxfehler in Direktiven}

\section{Compiler}
Der Compiler übersetzt den vorverarbeiteten Quellcode in Zwischencode oder Maschinencode (z.B. in Objektdateien .o oder .obj).
\subsection{Typische build Fehler}
\textit{Syntaxfehler:} fehlendes Semikolon, falsche Klammern\\
\textit{Typfehler:} int = char* ohne Cast\\
\textit{Unzulässige Operationen:} Division durch 0 (wenn zur Compilezeit berechenbar)\\
\textit{Nicht deklarierte Variablen oder Funktionen}\\
\textit{Mehrdeutige oder widersprüchliche Deklarationen}\\


\section{Linker}
Der Linker nimmt die vom Compiler erzeugten Objektdateien und verbindet sie zu einer ausführbaren Datei. Dabei löst er Symbolreferenzen auf, also z.B. wo eine Funktion oder Variable definiert ist.
Er fügt auch Standardbibliotheken und evtl. eigene Bibliotheken hinzu.

\subsection{Typische build Fehler}
\textit{Nicht definierte Symbole:} undefined reference to 'main' oder printf\\
\textit{Doppelt definierte Symbole:} multiple definition of 'myFunction'\\
\textit{Fehlende Objektdateien oder Bibliotheken}\\
\textit{Falsche Link-Reihenfolge}

\section{Modularisierung}
\begin{center}
\includegraphics[width = 0.6\linewidth]{Bilder/Modularisierung.png}  
\end{center}
\begin{enumerate}
    \item Gruppierung von zusammengehörigen Funktionen und Variabeln in einer Datei (strukturierter)
    \item Quelldateien können separat kompiliert werden → bei grossen Projekten Zeitersparnisse in der Entwicklung
    \item Funktionen lassen sich wiederverwenden
\end{enumerate}

\section{Structs}
Mit Structs können Daten zusammengefasst werden. Die Reihenfolge im Speicher entspricht der Reihenfolge der Deklarierung.
\begin{lstlisting}
struct tag_name_s {
    int a;
    int *b;
    float c;
} var = {.a = 5, .b = &(var.a)};
\end{lstlisting}
Ohne Typedef können zwischen $\}$ und $;$ Variablen deklariert und initialisiert werden. Variablen können ebenfals wie folgt deklariert und initialisiert werden:
\begin{lstlisting}
struct tag_name_s var2 = {.a = 7, .c = 1.23};
struct tag_name_s var3 = var2; //koppiert struct
\end{lstlisting}

\subsection{Pointer auf Structs}
\begin{lstlisting}
struct tag_name_s * ptr_s = &var2;
int x = ptr_s -> a; // x = 7
float y = (*ptr_s).c; // y = 1.23
\end{lstlisting}

\section{Unions}
Gleich wie Struct: \texttt{=} Operator zum kopieren und können über \texttt{return} als Rückgabewert einer Funktion fungieren. Der Unterschied ist jedoch, dass die Daten an der selben Speicheradresse stehen. Aus diesem Grund kann nur immer ein Mitglied der Union initalisert werden.
\begin{lstlisting}
union tag_name_u{
    uint16_t a;
    struct {
        uint8_t b;
        uint8_t c;    
    }byte;
}var = {.a = 10};
var.byte.c = 1;     // beschreiben eines bytes
printf("%d\n",var.byte.b);  // = 10
\end{lstlisting}
Achtung! Beim zugreifen auf ein byte muss berücksichtigt werden wie gespeichert wird (little endian, kleines ende zuerst).

\section{Bitfelder}
Bitfelder können mittles Struct oder Union gemacht werden. Durch Standard garantierte Typen: \_Bool, int, unsigned int, signed int Bitfelder haben im üblichen Sinne keine Adressen $\rightarrow$ direkt speichern in Bitfeld mit scanf ist nicht erlaubt $\rightarrow$ keine Adressoperatoren wie \& möglich
\begin{lstlisting}
struct file_date {
uint16_t day: 5;    // LSB
uint16_t month: 4;
uint16_t year: 7;   // MSB
};
struct file_date fd;
fd.day = 28;
fd.month = 12;
fd.year = 8; /* represents 1988 */

struct file_time {
uint16_t: 5; /* not used */
uint16_t minutes: 6;
uint16_t hours: 5;
} u = {.bits ={.minutes=1, .hours =1} };
struct s {
uint16_t a: 4;
uint16_t : 0; /* 0-length bit-field */
uint16_t b: 8;
}; 
// 0 gibt compiler an -> darauffolgende Element am Anfang einer Sepichereinheit (32 bit -> beginn einer 32 bit auch wenn nur 8 bit)
\end{lstlisting}
\section{Ausrichtungsanforderung}
Datentyp müss auf einer Speicheradresse leigen, die ein Vielfaches von sich selbst ist\\
int32\_t $\rightarrow$ 4 Bytes $\rightarrow$ Adresse durch vier Teilbar $\rightarrow$ 0x100, 0x104, ... 2 Bytes $\rightarrow$ 0x100, 0x102, 0x104, ...\\
\textbf{word:} natürliche Verarbeitungsbreite (in Bytes) der CPU
typischerweise 32 Bit oder 64 Bit
\subsection{Padding (16bit)}
Compiler fügt zusätzliche unbenannte Datenelemente ein
kann auch am Ende einer Struktur passieren um diese abzuschliessen $\rightarrow$ immer der Grösse nach ordnen (ab- oder aufsteigend)
\begin{lstlisting}
struct s2 {
char a;
uint16_t b;
}
\end{lstlisting}
\begin{center}
\includegraphics[width = \linewidth]{Bilder/Bitfelder_Struct_loecher.png}   
\end{center}
    


\section{Verkettete Listen}
\includegraphics[width = 0.4\linewidth]{Bilder/Verkettere_listen_Grafisch.png}
\subsection{Konten}
\begin{lstlisting}
struct node_s{
    int value;
    struct node_s *next;
}
struct node_s *first = NULL;
\end{lstlisting}
\subsection{Knoten erstellen}
\begin{lstlisting}
struct node_s *new_node;
new_node = malloc(sizeof(struct node_s));

new_node->value = 10;
\end{lstlisting}
\subsection{Knoten einfügen}
\begin{lstlisting}
first = add_to_list(first,10);
\end{lstlisting}
\begin{lstlisting}
struct node_s *add_to_list(struct node_s *list, int n){
    struct node_s *new_node;
    new_node = malloc(sizeof(struct node_s));
    if(new_node == NULL) {
        printf("Error:no memory available\n");
        exit( EXIT_FAILURE );
    }
    new_node ->value = n;
    new_node ->next = list;
    return
    new_node;
}
\end{lstlisting}

\vfill\null
\columnbreak

\subsection{Liste durchsuchen}
\begin{lstlisting}
struct node_s *search_list(struct node_s *list, int n) {
    struct node_s *p;
    for(p = list; p != NULL; p = p->next) {
        if(p->value == n) {
            return p;
        }
    }
    return NULL;
}
\end{lstlisting}
\subsection{Knoten löschen}
\begin{lstlisting}
struct node_s *delete_from_list(struct node_s *list, int n){
    struct node_s *cur, *prev;

    for(cur = list, prev = NULL;
        cur != NULL && cur->value != n;
        prev = cur, cur = cur->next);
    if(cur == NULL)    //n not found
        retrun list;
    if(prev == NULL)
        list == list->next;
    else
        prev->next = cur->next;
    free(cur);
    return list;
}
\end{lstlisting}
\section{Reaktive Systeme}
System, dass auf äussere in beliebiger Reihenfolge auftretende Ereignisse, innerhalb einer gewissen Zeit, durch vorgegebenes Verhalten reagiert\\
\begin{itemize}
    \item Nicht-blockierende Erkennung von Ereignis-Typen $\rightarrow$ Funktion, z.B. check\_timeout\_expired
    \item Event Handler (aka Callbacks) für die Verarbeitung von Events $\rightarrow$ Funktionszeiger, z.B. handle\_timeout, aufgerufen aus check\_timeout\_expired
    \item Mainloop für die Disposition (engl. Dispatching): Ereignis $\rightarrow$ Event-Handler (Reaktion) $\rightarrow$ erkennt Ereignisse, führt Event-Handler aus
\end{itemize}
\includegraphics[width = \linewidth]{Bilder/FSM.png}

\vfill\null
\columnbreak

\subsection{Expliziter Mainloop (Superloop)}
Die main-Funktion läuft immer innerhalb einer Unendlichschleife werden Ereignisse abgehandelt
$\rightarrow$ dadurch kann blockierender Code vermieden werden\\
Reaktion auf ein Ereignis abhängig von: Ereignis-Typ und Kontext (Zustand)\\
Alternative zu Superloop: Multi-Threading\\
\includegraphics[width = \linewidth]{Bilder/expliziter_Mainloop.png}

\subsection{Event-Handler}
\includegraphics[width = \linewidth]{Bilder/Event_Handler.png}

\subsection{Zustandsübergangstabelle}
\includegraphics[width = \linewidth]{Bilder/Zustandstabelle.png}


\section{Mainloop}
Klassischer Mainloop aufbau.
\begin{lstlisting}
/* main.c */
#include "main_loop.h"
//more includes...
//defines...
int main(void){
    init(); //board periphery
    timer_init( handle_timeout ); //callback
    //weitere Initalisierungen...
    
    main_loop();    //never leave
    return 0;
}
\end{lstlisting}

\vfill\null
\columnbreak

\begin{lstlisting}
/* main_loop.c */
static void (*timeout_cb) (void) = NULL;
void main_loop(void) {
    while(true) {
      if(timer_expired() && timeout_cb !=NULL)
        timeout_cb(); //invoke callback
    }
}
\end{lstlisting}

\section{Superloop Template}
\subsection{main.c}
\begin{lstlisting}
#include <stdio.h>
#include <stdint.h>
#include <string.h>
#include "board.h"
#include "main_loop.h"
#include "timer.h"
#include "buttons.h"

void handle_timeout() {
    printf("handle_timeout called!\n");
}
void handle_button(uint8_t btn_no, bool b) {
    printf("handle_button(sw%u,%s) called!\n",
            btn_no, b? "true":"false");
}
int main(void) {
    init(); //initialize board periphery
    timer_init( handle_timeout );
    button_init( 0, handle_button );
    button_init( 1, handle_button );
    button_init( 2, handle_button );
    
    main_loop(); //never returns
    return 0;
}
\end{lstlisting}
\subsection{timer.h}
\begin{lstlisting}
#ifndef TIMER_H_
#define TIMER_H_

//returns true if timestamp a is before b
#define BEFORE_U32(a, b) ((uint32_t)((a)-(b)) > 0x80000000u)
//returns true if timestamp a is after b
#define AFTER_U32(a, b) ((uint32_t)((a)-(b)) < 0x80000000u)

//timer callback init function 
void timer_init( void (*cb) (void) );

// function called in main_loop 
// and then to call its event handler
void check_timer_expired( void );

#endif /* TIMER_H_ */
\end{lstlisting}
\subsection{timer.c}
\begin{lstlisting}
#include <stdio.h>
#include "board.h"
#include "timer.h"

static uint32_t next_timeout = 0;
// function pointer to store timeout callback
static void (*timeout_cb) (void) = NULL;

void timer_init( void (*cb) (void) ) {
    next_timeout = 0;
    timeout_cb = cb;
}
void check_timer_expired( void ) {
    // if ( (int32_t)(time_get_ms() - next_timeout) >= 0 ) {
    if(AFTER_U32(time_get_ms(),next_timeout)){
        next_timeout = time_get_ms() + 1000;
        if( timeout_cb != NULL )
            timeout_cb(); //invoke callback
    }
}
\end{lstlisting}
\subsection{main\_loop.h}
\begin{lstlisting}
#ifndef MAIN_LOOP_H_
#define MAIN_LOOP_H_
void main_loop(void);
#endif /* MAIN_LOOP_H_ */
\end{lstlisting}
\subsection{main\_loop.c}
\begin{lstlisting}
#include <stdbool.h>
#include "main_loop.h"
#include "timer.h"
#include "buttons.h"

void main_loop(void) {
    while(true) {
        check_timer_expired();
        check_buttons();
    }
}
\end{lstlisting}


\section{Anhang}
\subsection{string.h}
\includegraphics[width = \linewidth]{Bilder/String_h.png}

\subsection{Operatoren}
\includegraphics[width = \linewidth]{Bilder/Operatoren.png}

\subsection{Last Advice}
\begin{center}
\includegraphics[width = 0.8\linewidth]{Bilder/Writing_code.jpg}
\end{center}



\end{multicols*}
\end{document}